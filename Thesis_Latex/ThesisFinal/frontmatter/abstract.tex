%!TEX root = ../dissertation.tex
% the abstract
Inserted in the ATLASCAR2 project this work aims to develop a short-term path planning framework for driver assistance in dynamic environments. In order to achieve this objective, it was made a preliminary study of the existing local path planning methods and the projects that have already been developed in this field, their advantages and disadvantages were weighed and the most successful approaches applied to local navigation in real autonomous driving projects were taken into account.
This thesis presents two different strategies for a self-driving car short-term path planning among multiple moving obstacles. The main task is to study and implement a motion planning and execution framework in order to make ATLASCAR2 coexist with other moving obstacle vehicles by avoiding collision and overtake them when necessary and possible. The first method developted, is an obstacle avoidance system that moves the vehicle around different moving obstacles while the second algorithm is a lane following system that keeps the ATLASCAR2 traveling along the centerline of the lanes on the road. The proposed techniques, based on the adaptive Model Predictive Control paradigm, solve optimization problems formulated in terms of cost minimization under constraints. Simulation results, developted in a MATLAB/Simulink enviroment, demonstrate and verify the feasibility and the usefulness of methods considering different scenarios, opening space for real scenario implementation.
